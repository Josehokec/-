\problem{证明: 从本原函数出发, 经复合和算子$\displaystyle\prod_{i=n}^m[\cdot]$可以生成所有的初等函数. 这里$$\prod_{i=n}^m [f(i)] = 
    \protect\begin{cases}
        f(n)\cdot f(n+1)\cdot\quad\cdots\quad\cdot f(m), & \textrm{若}m\geqslant n,\\
        1, & \textrm{若}m<n.
    \protect\end{cases} $$}
\begin{proof}
    由初等函数的定义和引理1.12, 只需要证明$\ddot{-}$, 有界迭加算子$\Sigma[\cdot]$和有界迭乘算子$\Pi[\cdot]$能用复合以及算子$\displaystyle\prod_{i=n}^m[\cdot]$表示.

    有界迭乘算子是算子$\displaystyle\prod_{i=n}^m[\cdot]$中$n=0$的特例.

    首先不难构造出以下几个函数:

    指数函数\footnote{这里约定$0^0=1$.}$$x^y=\prod_{i=1}^y[P_1^1(x)]$$
	
    取反函数$$N(x)=\prod_{i=1}^x[Z(i)]$$
	
    $x\leqslant y$的特征函数$$\mathrm{leq}(x,y)=\prod_{i=x}^y[Z(i)]$$
	
    $x\geqslant y$的特征函数$$\mathrm{geq}(x,y)=\prod_{i=y}^x[Z(i)]$$

    然后:

    $x=y$的特征函数$$\eq(x,y)=\mathrm{leq}(x,y)^{N(\mathrm{geq}(x,y))}$$
    $$2^x=\prod_{i=1}^x[S\circ S\circ Z(i)]$$
    
    令$$\log(x)=\begin{cases}
        \log_2x,&\textrm{若}\log_2x\in\mathbb{N},\\
        1,&\textrm{否则.}
    \end{cases}$$
	
    那么
    $$\log(x)=\prod_{i=0}^x[i^{N(\eq(2^i,x))}]$$
    $$\log(2^x)=\prod_{i=0}^{2^x}[i^{N(\eq(2^i,2^x))}]$$

    所以
    $$\sum_{i=n}^m f(i,\vec{x})=\log(2^{\sum_{i=n}^m f(i,\vec{x})})=\log(\prod_{i=n}^m 2^{f(i,\vec{x})})$$

    $$x\ddot{-}y=\sum_{i=y+1}^x[S\circ Z(i)] + \sum_{i=x+1}^y[S\circ Z(i)]$$

    综上, 证明完毕.
\end{proof}