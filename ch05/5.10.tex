\problem{设$f:\mathbb{N}\to\mathbb{N}$定义如下:
$$\protect\begin{aligned}
f(0)&=0,\\
f(x+1)&=g(f(x)).
\protect\end{aligned}$$
证明: 若$g$为Turing-可计算, 则$f$为Turing-可计算.}
\begin{proof}
按照定理5.14的证明方式, 令$y$恒为0. 因此首先在原来的输入上添加0作为$y$, 构建机器$M_1$如表\ref{tab:sol5.10m1}:

\begin{table}[!htbp]
\centering
\caption{题5.10机器$M_1$}
\label{tab:sol5.10m1}
\begin{tabularx}{\textwidth}{Y|Y|Y}
\thickhline
    &  0    &      1   \\
\hline
1   & $0R2$ &   $1R1$   \\
\hline
2   & $1L3$ &           \\
\hline
3   & $0L4$ &           \\
\hline
4   & $0R5$ &   $1L4$   \\
\thickhline
\end{tabularx}
\end{table}

易知$M_1|1:0\underset{\uparrow}{1}^{x+1}0\cdots\twoheadrightarrow5:0\underset{\uparrow}{1}^{x+1}010\cdots$.

由于$g$是Turing-可计算的, 所以存在机器$\machine{g}$计算函数$g$.

构造机器$M_2$如表\ref{tab:sol5.10m2}所示:
\begin{table}[!htbp]
\centering
\caption{题5.4机器$M_2$}
\label{tab:sol5.10m2}
\begin{tabularx}{\textwidth}{Y|Y|Y}
\thickhline
    &  0    &      1   \\
\hline
1   &       &   $0R2$   \\
\hline
2   & $0Ru$ &   $1R3$   \\
\hline
3   & $0R4$ &   $1R3$   \\
\thickhline
\end{tabularx}
\end{table}

令$M_3=M_2\concat \machine{g}+3 \concat \machine{compress} \concat \machine{shiftl}$, 并令$\machine{f}=\mathrm{repeat}M_3$, $M=M_1\concat \machine{f}$即为能计算$f$的机器, 因此$f$为Turing-可计算.
\end{proof}