\problem{证明: 对于任何$M,N\in\Lambda$, 若$M\equivbeta N$, 则存在$T$使$M\manystepbeta T$且$N\manystepbeta T$. 这就是对于$\equivbeta$的CR性质.}
\begin{proof}{\color {red} {对第一版答案做出了补充说明.}}

    \textbf{方法一}:由习题3.9知:$\exists P_0, \cdots , P_n$使$M \equiv P_0, N \equiv P_n$且$\forall i < n$有$P_i \rightarrow_{\beta} P_{i+1}$或$P_i \leftarrow_{\beta} P_{i+1}$.
    
    下面证明:
    $$\exists T_i \  s.t. \  P_0 \twoheadrightarrow_{\beta} T_i \ and \  P_i \twoheadrightarrow_{\beta} T_i \quad(\star)$$
    
    对$i$做归纳证明.
    
    1) 当$i=0$时,取$T_0$为$M$即可,此时$(\star)$成立;
    
    2) 假设$i=k$时,$\exists T_k \  s.t. \  P_0 \twoheadrightarrow_{\beta} T_k \ and \  P_k \twoheadrightarrow_{\beta} T_k $;
    
    3) 当$i=k+1<n$时,有$P_0 \twoheadrightarrow_{\beta} T_k \ and \  P_k \twoheadrightarrow_{\beta} T_k$(归纳假设),则
    
        情况1:$P_k \twoheadrightarrow_{\beta} P_{k+1}$,从而由CR性质,$\exists T_{k+1} \ s.t. \  T_k \twoheadrightarrow_{\beta} T_{k+1} \ and \ P_{k+1} \twoheadrightarrow_{\beta} T_{k+1}$,从而$(\star)$成立;
    
        情况2:$P_k \leftarrow_{\beta} P_{k+1}$,取$T_{k+1}$为$P_k$即可,此时$(\star)$也成立.
    
    于是归纳完成.
    
    因此有$\exists T_n \  s.t. \  P_0 \twoheadrightarrow_{\beta} T_n \ and \  P_n \twoheadrightarrow_{\beta} T_n $,取$T$为$T_n$即有$M\twoheadrightarrow_{\beta} T \ and \ N \twoheadrightarrow_{\beta} T$.
    
    %换行
    \hspace*{\fill}
    
    \textbf{方法二}:
    $M\equivbeta N$蕴含
	$$(M,N)\in\bigcup_{i\in\mathbb{N}}(\to_{\beta}\cup\leftarrow_{\beta})^k.$$
	
	当$k=0$时, $M\equivbeta N$.
	假设对所有$(M,N)\in(\to_{\beta}\cup\leftarrow_{\beta})^k$, 存在$T\in\Lambda$使得$M\manystepbeta T$且$N\manystepbeta T$.
	
	那么当$(M,N)\in(\to_{\beta}\cup\leftarrow_{\beta})^{k+1}$时, 要么有$M\onestepbeta P=_\beta N$, 要么有$M\leftarrow_\beta P=_\beta N$, 其中$(P,N)\in(\to_{\beta}\cup\leftarrow_{\beta})^k$. 那么存在$T_0$使得$P\manystepbeta T_0$且$N\manystepbeta T_0$. 由于$\manystepbeta$是传递的, 所以$M\manystepbeta T_0$.
	
	由于$\manystepbeta$的CR性质, $P\manystepbeta M$且$P\manystepbeta T_0$可得到, 存在$T\in\Lambda$使得$M\manystepbeta T$且$T_0\manystepbeta T$. 由于$\manystepbeta$是传递的, 有$N\manystepbeta T_0$且$T_0\manystepbeta T$, 因此$N\manystepbeta T$.
	
	因此对所有的$k\in\mathbb{N}$, 这样的$T$都存在. 命题成立.
\end{proof}